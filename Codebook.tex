\documentclass[a4paper,10pt,twocolumn,oneside]{article}
\setlength{\columnsep}{10pt}                                                                    %兩欄模式的間距
\setlength{\columnseprule}{0pt}                                                                %兩欄模式間格線粗細

\usepackage{amsthm}								%定義,例題
\usepackage{amssymb}
%\usepackage[margin=2cm]{geometry}
\usepackage{fontspec}								%設定字體
\usepackage{color}
\usepackage[x11names]{xcolor}
\usepackage{listings}								%顯示code用的
\usepackage[Glenn]{fncychap}						%排版,頁面模板
\usepackage{fancyhdr}								%設定頁首頁尾
\usepackage{graphicx}								%Graphic
\usepackage{enumerate}
\usepackage{changepage}
\usepackage{titlesec}
\usepackage{amsmath}
\usepackage{codebook}
\usepackage[CheckSingle, CJKmath]{xeCJK}
% \usepackage{CJKulem}

%\usepackage[T1]{fontenc}
\usepackage{amsmath, courier, listings, fancyhdr, graphicx}
\topmargin=0pt
\headsep=5pt
\textheight=780pt
\footskip=0pt
\voffset=-40pt
\textwidth=545pt
\marginparsep=0pt
\marginparwidth=0pt
\marginparpush=0pt
\oddsidemargin=0pt
\evensidemargin=0pt
\hoffset=-42pt

%\renewcommand\listfigurename{圖目錄}
%\renewcommand\listtablename{表目錄} 

%%%%%%%%%%%%%%%%%%%%%%%%%%%%%

%\setmainfont{Consolas}				%主要字型
\setCJKmainfont{msjh.ttc}			%中文字型
%\setmainfont{Linux Libertine G}
\setmonofont{consola.ttf}
%\setmainfont{sourcecodepro}
\XeTeXlinebreaklocale "zh"						%中文自動換行
\XeTeXlinebreakskip = 0pt plus 1pt				%設定段落之間的距離
\setcounter{secnumdepth}{3}						%目錄顯示第三層

%%%%%%%%%%%%%%%%%%%%%%%%%%%%%
\makeatletter
\lst@CCPutMacro\lst@ProcessOther {"2D}{\lst@ttfamily{-{}}{-{}}}
\@empty\z@\@empty
\makeatother
\lstset{											% Code顯示
language=C++,										% the language of the code
basicstyle=\ttfamily, 						% the size of the fonts that are used for the code
%numbers=left,										% where to put the line-numbers
numberstyle=\footnotesize,						% the size of the fonts that are used for the line-numbers
stepnumber=1,										% the step between two line-numbers. If it's 1, each line  will be numbered
numbersep=5pt,										% how far the line-numbers are from the code
backgroundcolor=\color{white},					% choose the background color. You must add \usepackage{color}
showspaces=false,									% show spaces adding particular underscores
showstringspaces=false,							% underline spaces within strings
showtabs=false,									% show tabs within strings adding particular underscores
frame=false,											% adds a frame around the code
tabsize=2,											% sets default tabsize to 2 spaces
captionpos=b,										% sets the caption-position to bottom
breaklines=true,									% sets automatic line breaking
breakatwhitespace=false,							% sets if automatic breaks should only happen at whitespace
escapeinside={\%*}{*)},							% if you want to add a comment within your code
morekeywords={*},									% if you want to add more keywords to the set
keywordstyle=\bfseries\color{Blue1},
commentstyle=\itshape\color{Red4},
stringstyle=\itshape\color{Green4}
}



%%%%%%%%%%%%%%%%%%%%%%%%%%%%%

\begin{document}
\pagestyle{fancy}
\fancyfoot{}
%\fancyfoot[R]{\includegraphics[width=20pt]{ironwood.jpg}}
\fancyhead[L]{NCNU - No idea codebook}
\fancyhead[R]{\thepage}
\renewcommand{\headrulewidth}{0.4pt}
\renewcommand{\contentsname}{Contents} 
\scriptsize
\begingroup
\let\clearpage\relax
% \tableofcontents
\endgroup
\tableofcontents
%%%%%%%%%%%%%%%%%%%%%%%%%%%%%
% \newpage

\section{Basic}
  \code{Basic codeblock setting}{./Basic/Basic codeblock setting.cpp}
  \code{Basic vim setting}{./Basic/Basic vim setting.cpp}
  \code{Code Template}{./Basic/Code Template.cpp}
  \code{Python}{./Basic/Python.cpp}
  \code{Range data}{./Basic/Range data.cpp}
  \code{Some Function}{./Basic/Some Function.cpp}
  \code{Time}{./Basic/Time.cpp}

\section{DP}
  \code{3維DP思路}{./DP/3維DP思路.cpp}
  \code{Knapsack Bounded}{./DP/Knapsack Bounded.cpp}
  \code{Knapsack sample}{./DP/Knapsack sample.cpp}
  \code{Knapsack Unbounded}{./DP/Knapsack Unbounded.cpp}
  \code{LCIS}{./DP/LCIS.cpp}
  \code{LCS}{./DP/LCS.cpp}
  \code{LIS O(Nlog(N))}{./DP/LIS O(Nlog(N)).cpp}
  \code{LIS}{./DP/LIS.cpp}
  \code{LPS}{./DP/LPS.cpp}
  \code{Max\_subarray}{./DP/Max_subarray.cpp}
  \code{Money problem}{./DP/Money problem.cpp}

\section{Flow \& matching}
  \code{Dinic}{./Flow & matching/Dinic.cpp}
  \code{Edmonds\_karp}{./Flow & matching/Edmonds_karp.cpp}
  \code{hungarian}{./Flow & matching/hungarian.cpp}
  \code{Maximum\_matching}{./Flow & matching/Maximum_matching.cpp}
  \code{MFlow Model}{./Flow & matching/MFlow Model.cpp}

\section{Geometry}
  \code{Circle Intersect}{./Geometry/Circle Intersect.cpp}
  \code{Closest Pair}{./Geometry/Closest Pair.cpp}
  \code{Line}{./Geometry/Line.cpp}
  \code{max\_cover\_rectangle}{./Geometry/max_cover_rectangle.cpp}
  \code{Point}{./Geometry/Point.cpp}
  \code{Polygon}{./Geometry/Polygon.cpp}
  \code{Triangle}{./Geometry/Triangle.cpp}

\section{Graph}
  \code{Bellman-Ford}{./Graph/Bellman-Ford.cpp}
  \code{BFS-queue}{./Graph/BFS-queue.cpp}
  \code{DFS-rec}{./Graph/DFS-rec.cpp}
  \code{Dijkstra}{./Graph/Dijkstra.cpp}
  \code{Euler circuit}{./Graph/Euler circuit.cpp}
  \code{Floyd-warshall}{./Graph/Floyd-warshall.cpp}
  \code{Hamilton\_cycle}{./Graph/Hamilton_cycle.cpp}
  \code{Kruskal}{./Graph/Kruskal.cpp}
  \code{Minimum Weight Cycle}{./Graph/Minimum Weight Cycle.cpp}
  \code{Prim}{./Graph/Prim.cpp}
  \code{Union\_find}{./Graph/Union_find.cpp}

\section{Mathematics}
  \tex{Catalan}{./Mathematics/Catalan.tex}
  \code{Combination}{./Mathematics/Combination.cpp}
  \code{Extended Euclidean}{./Mathematics/Extended Euclidean.cpp}
  \tex{Fermat}{./Mathematics/Fermat.tex}
  \code{Hex to Dec}{./Mathematics/Hex to Dec.cpp}
  \code{Log}{./Mathematics/Log.cpp}
  \tex{Mod性質}{./Mathematics/Mod性質.tex}
  \code{PI}{./Mathematics/PI.cpp}
  \code{Prime table}{./Mathematics/Prime table.cpp}
  \code{Prime判斷}{./Mathematics/Prime判斷.cpp}
  \code{Round(小數)}{./Mathematics/Round(小數).cpp}
  \code{二分逼近法}{./Mathematics/二分逼近法.cpp}
  \tex{公式}{./Mathematics/公式.tex}
  \code{四則運算}{./Mathematics/四則運算.cpp}
  \code{因數表}{./Mathematics/因數表.cpp}
  \code{數字乘法組合}{./Mathematics/數字乘法組合.cpp}
  \code{數字加法組合}{./Mathematics/數字加法組合.cpp}
  \code{羅馬數字}{./Mathematics/羅馬數字.cpp}
  \code{質因數分解}{./Mathematics/質因數分解.cpp}
  \code{質數數量}{./Mathematics/質數數量.cpp}

\section{Other}
  \code{binary search 三類變化}{./Other/Binary search 三類變化.cpp}
  \code{Heap sort}{./Other/Heap sort.cpp}
  \code{Josephus}{./Other/Josephus.cpp}
  \code{Merge sort}{./Other/Merge sort.cpp}
  \code{Quick sort}{./Other/Quick sort.cpp}
  \code{Weighted Job Scheduling}{./Other/Weighted Job Scheduling.cpp}
  \code{多區間算最大}{./Other/Largest Multi-interval.cpp}
  \code{數獨解法}{./Other/Sudoku solution.cpp}

\section{String}
  \code{KMP}{./String/KMP.cpp}
  \code{Min Edit Distance}{./String/Min Edit Distance.cpp}
  \code{Sliding window}{./String/Sliding window.cpp}
  \code{Split}{./String/Split.cpp}

\section{data structure}
  \code{Bigint}{./data structure/Bigint.cpp}
  \code{DisjointSet}{./data structure/DisjointSet.cpp}
  \code{Matirx}{./data structure/Matirx.cpp}
  \code{Trie}{./data structure/Trie.cpp}
  \code{分數}{./data structure/分數.cpp}


\end{document}
