\documentclass[a4paper,10pt,oneside]{article}
\setlength{\columnsep}{15pt}    %兩欄模式的間距
\setlength{\columnseprule}{0pt}

\usepackage[landscape]{geometry}
\usepackage{amsthm}								%定義,例題
\usepackage{amssymb}
\usepackage{fontspec}								%設定字體
\usepackage{color}
\usepackage[x11names]{xcolor}
\usepackage{xeCJK}								%xeCJK
\usepackage{listings}								%顯示code用的
%\usepackage[Glenn]{fncychap}						%排版,頁面模板
\usepackage{fancyhdr}								%設定頁首頁尾
\usepackage{graphicx}								%Graphic
\usepackage{enumerate}
\usepackage{titlesec}
\usepackage{amsmath}
\usepackage{pdfpages}
\usepackage{multicol}
\usepackage{fancyhdr}
%\usepackage[T1]{fontenc}
\usepackage{amsmath, courier, listings, fancyhdr, graphicx}

%\topmargin=0pt
%\headsep=5pt
\textheight=530pt
%\footskip=0pt
\voffset=-20pt
\textwidth=800pt
%\marginparsep=0pt
%\marginparwidth=0pt
%\marginparpush=0pt
%\oddsidemargin=0pt
%\evensidemargin=0pt
\hoffset=-100pt

%\setmainfont{Consolas}				%主要字型
\setCJKmainfont{msjh.ttc}			%中文字型
%\setmainfont{Linux Libertine G}
\setmonofont{consola.ttf}
%\setmainfont{sourcecodepro}
\XeTeXlinebreaklocale "zh"						%中文自動換行
\XeTeXlinebreakskip = 0pt plus 1pt				%設定段落之間的距離
\setcounter{secnumdepth}{3}						%目錄顯示第三層

\makeatletter
\lst@CCPutMacro\lst@ProcessOther {"2D}{\lst@ttfamily{-{}}{-{}}}
\@empty\z@\@empty
\makeatother
\lstset{											% Code顯示
language=C++,										% the language of the code
basicstyle=\scriptsize\ttfamily, 						% the size of the fonts that are used for the code
numbers=left,										% where to put the line-numbers
numberstyle=\tiny,						% the size of the fonts that are used for the line-numbers
stepnumber=1,										% the step between two line-numbers. If it's 1, each line  will be numbered
numbersep=5pt,										% how far the line-numbers are from the code
backgroundcolor=\color{white},					% choose the background color. You must add \usepackage{color}
showspaces=false,									% show spaces adding particular underscores
showstringspaces=false,							% underline spaces within strings
showtabs=false,									% show tabs within strings adding particular underscores
frame=false,											% adds a frame around the code
tabsize=2,											% sets default tabsize to 2 spaces
captionpos=b,										% sets the caption-position to bottom
breaklines=true,									% sets automatic line breaking
breakatwhitespace=false,							% sets if automatic breaks should only happen at whitespace
escapeinside={\%*}{*)},							% if you want to add a comment within your code
morekeywords={*},									% if you want to add more keywords to the set
keywordstyle=\bfseries\color{Blue1},
commentstyle=\itshape\color{Red4},
stringstyle=\itshape\color{Green4},
}


\newcommand{\includecpp}[2]{
  \subsection{#1}
    \lstinputlisting{#2}
}

\newcommand{\includetex}[2]{
  \subsection{#1}
    \input{#2}
}


\begin{document}
  \begin{multicols}{4}
  \pagestyle{fancy}
  
  \fancyfoot{}
  \fancyhead[L]{NCNU - No idea codebook}
  \fancyhead[R]{\thepage}
  
  \renewcommand{\headrulewidth}{0.4pt}
  \renewcommand{\contentsname}{Contents}

   
  \scriptsize
  \section{Basic}
  \includecpp{Code Template}{./Basic/Code Template.cpp}
  \includecpp{Codeblock setting}{./Basic/Codeblock setting.cpp}
  \includecpp{IO\_fast}{./Basic/IO_fast.cpp}
  \includecpp{Python}{./Basic/Python.cpp}
  \includecpp{Range data}{./Basic/Range data.cpp}
  \includecpp{Some Function}{./Basic/Some Function.cpp}
  \includecpp{Time}{./Basic/Time.cpp}
  \includecpp{Vim setting}{./Basic/Vim setting.cpp}
\section{DP}
  \includecpp{3維DP思路}{./DP/3維DP思路.cpp}
  \includecpp{Knapsack Bounded}{./DP/Knapsack Bounded.cpp}
  \includecpp{Knapsack sample}{./DP/Knapsack sample.cpp}
  \includecpp{Knapsack Unbounded}{./DP/Knapsack Unbounded.cpp}
  \includecpp{LCIS}{./DP/LCIS.cpp}
  \includecpp{LCS}{./DP/LCS.cpp}
  \includecpp{LIS}{./DP/LIS.cpp}
  \includecpp{LPS}{./DP/LPS.cpp}
  \includecpp{Max\_subarray}{./DP/Max_subarray.cpp}
  \includecpp{Money problem}{./DP/Money problem.cpp}
\section{Flow \& matching}
  \includecpp{Dinic}{./Flow & matching/Dinic.cpp}
  \includecpp{Edmonds\_karp}{./Flow & matching/Edmonds_karp.cpp}
  \includecpp{hungarian}{./Flow & matching/hungarian.cpp}
  \includecpp{Maximum\_matching}{./Flow & matching/Maximum_matching.cpp}
  \includecpp{MFlow Model}{./Flow & matching/MFlow Model.cpp}
\section{Geometry}
  \includecpp{Closest Pair}{./Geometry/Closest Pair.cpp}
  \includecpp{Line}{./Geometry/Line.cpp}
  \includecpp{Point}{./Geometry/Point.cpp}
  \includecpp{Polygon}{./Geometry/Polygon.cpp}
  \includecpp{Triangle}{./Geometry/Triangle.cpp}
\section{Graph}
  \includecpp{Bellman-Ford}{./Graph/Bellman-Ford.cpp}
  \includecpp{BFS-queue}{./Graph/BFS-queue.cpp}
  \includecpp{DFS-rec}{./Graph/DFS-rec.cpp}
  \includecpp{Dijkstra}{./Graph/Dijkstra.cpp}
  \includecpp{Euler circuit}{./Graph/Euler circuit.cpp}
  \includecpp{Floyd-warshall}{./Graph/Floyd-warshall.cpp}
  \includecpp{Hamilton\_cycle}{./Graph/Hamilton_cycle.cpp}
  \includecpp{Kruskal}{./Graph/Kruskal.cpp}
  \includecpp{Prim}{./Graph/Prim.cpp}
  \includecpp{Union\_find}{./Graph/Union_find.cpp}
\section{Mathematics}
  \includetex{Catalan}{./Mathematics/Catalan.tex}
  \includecpp{Combination}{./Mathematics/Combination.cpp}
  \includecpp{Extended Euclidean}{./Mathematics/Extended Euclidean.cpp}
  \includetex{Fermat}{./Mathematics/Fermat.tex}
  \includecpp{Hex to Dec}{./Mathematics/Hex to Dec.cpp}
  \includecpp{Log}{./Mathematics/Log.cpp}
  \includecpp{Mod}{./Mathematics/Mod.cpp}
  \includetex{Mod性質}{./Mathematics/Mod性質.tex}
  \includecpp{PI}{./Mathematics/PI.cpp}
  \includecpp{Prime table}{./Mathematics/Prime table.cpp}
  \includecpp{Prime判斷}{./Mathematics/Prime判斷.cpp}
  \includecpp{Round(小數)}{./Mathematics/Round(小數).cpp}
  \includecpp{二分逼近法}{./Mathematics/二分逼近法.cpp}
  \includetex{公式}{./Mathematics/公式.tex}
  \includecpp{四則運算}{./Mathematics/四則運算.cpp}
  \includecpp{因數表}{./Mathematics/因數表.cpp}
  \includecpp{數字乘法組合}{./Mathematics/數字乘法組合.cpp}
  \includecpp{數字加法組合}{./Mathematics/數字加法組合.cpp}
  \includecpp{羅馬數字}{./Mathematics/羅馬數字.cpp}
  \includecpp{質因數分解}{./Mathematics/質因數分解.cpp}
\section{Other}
  \includecpp{binary search 三類變化}{./Other/binary search 三類變化.cpp}
  \includecpp{heap sort}{./Other/heap sort.cpp}
  \includecpp{Merge sort}{./Other/Merge sort.cpp}
  \includecpp{Quick}{./Other/Quick.cpp}
  \includecpp{Weighted Job Scheduling}{./Other/Weighted Job Scheduling.cpp}
  \includecpp{數獨解法}{./Other/數獨解法.cpp}
\section{String}
  \includecpp{KMP}{./String/KMP.cpp}
  \includecpp{Min Edit Distance}{./String/Min Edit Distance.cpp}
  \includecpp{Sliding window}{./String/Sliding window.cpp}
  \includecpp{Split}{./String/Split.cpp}
\section{data structure}
  \includecpp{Bigint}{./data structure/Bigint.cpp}
  \includecpp{DisjointSet}{./data structure/DisjointSet.cpp}
  \includecpp{Matirx}{./data structure/Matirx.cpp}
  \includecpp{Trie}{./data structure/Trie.cpp}
  \includecpp{分數}{./data structure/分數.cpp}

  \clearpage
  \end{multicols}
  \newpage
  \begin{multicols}{4}
  \enlargethispage*{\baselineskip}
  \begin{center}
    \Huge\textsc{ACM ICPC Team Reference - Angry Crow Takes Flight!}
    \vspace{0.35cm}    
  \end{center}
  \tableofcontents
  \end{multicols}
  \clearpage
\end{document}
